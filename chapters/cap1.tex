\setchapterpreamble[u]{\margintoc}
\chapter{Cos'è un materiale metallico?}
\labch{cap1}
\index{metallurgia}
La metallurgia è la scienza e la tecnologia dei materiali metallici.
I materiali metallici si distinguono in:
\begin{itemize}
    \item \textbf{metalli puri}, rarissimi e di nessuna utilità, poiché hanno delle caratteristiche meccaniche bassissimi e scarse applicazioni;
    \item \textbf{leghe}, sono materiali con caratteristiche metalliche formati da più componenti, cioè da più metalli (es bronzo) o un metallo e da un non metallo (es acciaio), detto metalloide.
    \index{acciaio} \index{ghisa}
L’acciaio è la ferrolega più comune, formata da ferro e carbonio e in parti minori da manganese e silicio, già presenti nel minerale. Fino al 2\% di carbonio questa lega viene definita acciaio, al di sopra di tale valore si parlerà invece di ghisa. Tuttavia, aggiungendo altri elementi, si possono ottenere acciaio con tenori di carbonio superiori al 2\% e ghise con tenori di carbonio minori del 2\%. Nel campo aerospaziale, sono molto diffuse le leghe di alluminio, sia con metalli sia con non metalli (ex. lega di alluminio e silicio).
Non sempre l’unione di un metallo e di un non metallo dà origine a una lega, ma dipende dalle caratteristiche del prodotto finale, cioè da come si comporta tale materiale. Distinguiamo, infatti, le leghe dagli ossidi, materiali formati dall’unione di ossigeno O e di un metallo. Ad esempio, il ferro forma anche degli ossidi, come l’ematite $\mathrm{Fe_2O_3}$, la magnetite $\mathrm{Fe_3O_4}$ e la wustite $\mathrm{FeO}$. Altri ossidi importanti sono l’allumina $\mathrm{Al_2O_3}$, l’ossido di titanio $\mathrm{TiO_2}$ e l’ossido di calcio, o calce viva, $\mathrm{CaO}$.
Essi non possono essere considerati materiali metallici, in quanto privi delle caratteristiche che rendono tale un materiale metallico. Infatti, i materiali metallici hanno caratteristiche specifiche.
\end{itemize}