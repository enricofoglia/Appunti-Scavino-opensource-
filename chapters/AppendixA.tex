\setchapterstyle{lines}
\labpage{appendix}

\chapter{Quiz d'esame}

Sono successivamente riportate alcune domande proposte durante le sessioni d'esame online. Le risposte (inferite dalla non sempre trasparenti indicazioni del professore) sono riportate al finale.

\begin{enumerate}
    \item I convertitori hanno una capacità di:
    \begin{enumerate}
        \item 100 T
        \item 500 T
        \item 10 T
        \item 10000 T
        \item 1 T
    \end{enumerate}
    \item I convertitori LD hanno:
    \begin{enumerate}
        \item insufflaggio di aria e refrattario basico
        \item insufflaggio di aria e refrattario acido
        \item insufflaggio di azoto e refrattario basico
        \item insufflaggio di ossigeno e refrattario basico
        \item insufflaggio di ossigeno e refrattario acido
    \end{enumerate}
    \item Le leghe di alluminio:
    \begin{enumerate}
        \item si saldano più facilmente degli acciai
        \item hanno una conducibilità termica inferiore a quella degli acciai
        \item contengono rame per facilitare la saldatura
        \item il prezzo elevato è dovuto alla scarsità di materiale in natura
        \item contenenti rame non si saldano
    \end{enumerate}
    \item L’infragilimento da idrogeno
    \begin{enumerate}
        \item è causato dal raffreddamento in idrogeno
        \item è causato dalla bonifica in assenza di atmosfere protettive
        \item avviene in acciai con alte caratteristiche meccaniche
        \item è causato dal raffreddamento in acqua
        \item è causato dalla distensione a 180° per alcune ore
    \end{enumerate}
    \item La perlite è:
    \begin{enumerate}
        \item una miscela meccanica
        \item una soluzione solida di C in Fe
        \item composta da 11\% di austenite
        \item composta da 11\% di martensite
        \item una fase del diagramma di stato Fe-C
    \end{enumerate}
    \item Le soluzioni solide ordinate si trovano:
    \begin{enumerate}
        \item  sopra una temperatura critica
        \item a qualsiasi temperatura
        \item  sotto una temperatura critica
        \item sotto 0°C
        \item sopra 120°C
    \end{enumerate}
    \item A 0 K il numero di vacanze presente in una mole di Fe purissimo ha:
    \begin{enumerate}
        \item 0 vacanze 
        \item 4vacanze ogni 100000 atomi
        \item una vacanza ogni 10 atomi
        \item non è determinabile
        \item il numero di Avogadro di vacanze
    \end{enumerate}
    \item L’energia libera di formazione di un ossido metallico
    \begin{enumerate}
        \item aumenta all’aumentare della temperatura
        \item diminuisce all’aumentare della temperatura
        \item è indipendente dalla temperatura
        \item può crescere o diminuire in dipendenza del tipo di metallo all’aumentare della temperatura
        \item cresce, raggiunge un massimo e poi diminuisce all’aumentare della temperatura
    \end{enumerate}
    \item Mediamente l’altoforno si produce:
    \begin{enumerate}
        \item acciaio
        \item ghisa
        \item ghisa o acciaio in funzione del tenore di coke
        \item ghisa o acciaio in funzione della temperatura
        \item ghisa o acciaio in funzione del minerale prescelto
    \end{enumerate}
    \item Gli acciai dual phases sono costituiti da:
    \begin{enumerate}
        \item ferrite ed austenite
        \item ferrite ed austenite
        \item ferrite e perlite
        \item ferrite e martensite
        \item austenite residua e martensite
    \end{enumerate}
    \item Il microscopio metallografico
    \begin{enumerate}
        \item funziona per trasmissione
        \item permette l’analisi solo degli acciai
        \item permette l’analisi di qualsiasi superficie metallica
        \item funziona per riflessione
        \item permette l’analisi chimica delle superfici
    \end{enumerate}
    \item L’analisi al microscopio metallografico senza attacco metallografico permette di evidenzia:
    \begin{enumerate}
        \item i bordi di grano
        \item la distribuzione della perlite
        \item la distribuzione delle inclusioni 
        \item la forma delle lamelle di cementite e ferrite
        \item la presenza di austenite
    \end{enumerate}
    \item Il microscopio elettronico
    \begin{enumerate}
        \item non permette un analisi chimica puntuale
        \item richiede superfici metalliche perfettamente lucide
        \item il fascio di luce viene riflesso dal campione
        \item non da informazioni aggiuntive a quelle ottenibili con il microscopio ottico se non che l’immagine viene maggiormente ingrandita
        \item permette l’analisi delle superfici di frattura
    \end{enumerate}
    \item Quali elettroni sono responsabili delle proprietà magnetiche dei metalli:
    \begin{enumerate}
        \item elettroni disaccoppiati degli orbitali p
        \item elettroni accoppiati degli orbitali s
        \item elettroni disaccoppiati degli orbitali d
        \item elettroni accoppiati degli orbitali p
        \item elettroni accoppiati degli orbitali d
    \end{enumerate}
    \item Il rame è ferromagnetico:
    \begin{enumerate}
        \item Sopra il punto di Curie
        \item mai
        \item a temperatura ambiente
        \item sotto il punto di Curie
        \item a 0 K
    \end{enumerate}
    \item L’infragilimento da idrogeno:
    \begin{enumerate}
        \item si può evitare zincando i componenti
        \item si può evitare decappando i componenti
        \item si può evitare bonificando i pezzi
        \item si può evitare con una ricottura a 180°C per alcune ore
        \item si può evitare con un sottoraffreddamento
    \end{enumerate}
    \item Gli acciai a lavorabilità migliorata
    \begin{enumerate}
        \item contengono cromo
        \item contengono zolfo
        \item contengono zolfo e manganese
        \item sono bonificati
        \item sono solamente rinvenuti
    \end{enumerate}
    \item Gli acciai a lavorabilità migliorata:
    \begin{enumerate}
        \item possono contenere piombo
        \item sono stati sottoposti a ricottura a 180°C per alcune ore
        \item sono tutti acciai a basso tenore di carbonio
        \item contengono prevalentemente ferrite
        \item si temprano con raffreddamento in olio
    \end{enumerate}
    \item Negli acciai inossidabili la resistenza alla corrosione è dovuta:
    \begin{enumerate}
        \item alla presenza di nichel in tenori elevati
        \item ad alti tenori di carbonio
        \item alla presenza del cromo in tenori elevati
        \item alla presenza contemporaneamente di cromo e nichel in tenori elevati
        \item alla presenza contemporaneamente di cromo e carbonio in tenori elevati
    \end{enumerate}
    \item Nelle leghe di alluminio l’invecchiamento:
    \begin{enumerate}
        \item è causato da processi galvanici
        \item è dovuto ad un riscaldamento di solubilizzazione seguito d tempra e riscaldamento a 100-150°C
        \item avviene in leghe con basse caratteristiche meccaniche
        \item alla presenza contemporaneamente di cromo e nichel in tenori elevati
        \item alla presenza contemporaneamente di cromo e carbonio in tenori elevati
    \end{enumerate}
    \item La bonifica serve per:
    \begin{enumerate}
        \item aumentare la durezza
        \item diminuire la durezza
        \item aumentare la durezza e la tenacità
        \item aumentare la tenacità
        \item diminuire la tenacità
    \end{enumerate}
    \item La normalizzazione fornisce strutture:
    \begin{enumerate}
        \item più grossolane della ricottura
        \item più fini della tempra
        \item più fini della bonifica
        \item dipendenti dalla dimensione del componente
        \item essenzialmente perlitiche
    \end{enumerate}
    \item La normalizzazione è:
    \begin{enumerate}
        \item più costosa della tempra
        \item più costosa della ricottura
        \item più costosa della ricottura ma meno della tempra
        \item più costosa della tempra ma meno della bonifica
        \item meno costosa
    \end{enumerate}
    \item La struttura finale della bonifica è:
    \begin{enumerate}
        \item ferrite e perlite lamellare
        \item martensite
        \item martensite globulare
        \item ferrite e cementite fini e globulari
        \item essenzialmente perlitica
    \end{enumerate}
    \item La ferrite delta a 1492 °C ha una solubilità massima di carbonio di:
    \begin{enumerate}
        \item 2\%
        \item 0,08\%
        \item 0,18\%
        \item 0,1\%
        \item 4,3\%
    \end{enumerate}
    \item La trasformazione peritettica non interessa leghe Fe-C con tenore di carbonio:
    \begin{enumerate}
        \item inferiore a 0,18\%
        \item inferiore a 0,5\%
        \item superiore 0,02\%
        \item superiore 0,18\%
        \item superiore 0,5\%
    \end{enumerate}
    \item Per lavorare per asportazione di truciolo un componente conviene fare:
    \begin{enumerate}
        \item una ricottura
        \item una tempra
        \item una bonifica
        \item una normalizzazione
        \item un rinvenimento
    \end{enumerate}
    \item Un rinvenimento
    \begin{enumerate}
        \item segue la normalizzazione
        \item segue le ricottura
        \item precede la tempra
        \item precede la ricottura
        \item segue la tempra
    \end{enumerate}
    \item La fragilità da rinvenimento è dovuta a:
    \begin{enumerate}
        \item presenza di idrogeno
        \item presenza dello zolfo
        \item presenza del magnesio
        \item prematura precipitazione di carburi
        \item tardiva precipitazione di carburi
    \end{enumerate}
    \item La martensite è:
    \begin{enumerate}
        \item una fase del diagramma di stato Fe-C
        \item una soluzione solida di carbonio in ferrite metastabile a temperatura ambiente
        \item una soluzione solida di carbonio in forma stabile a temperatura ambiente
        \item una soluzione solida di carbonio in forma ad alta temperatura
        \item una soluzione solida di carbonio in forma a bassa temperatura
    \end{enumerate}
    \item Un acciaio con 0,4\% di carbonio, dopo ricottura, presenterà una struttura costituita da:
    \begin{enumerate}
        \item martensite fine
        \item martensite grossolana
        \item ferrite e perlite lamellare
        \item ferrite e perlite globulare
        \item ferrite globulare e perlite
    \end{enumerate}
    \item La presenza di elementi leganti negli acciai da bonifica:
    \begin{enumerate}
        \item stabilizza la martensite
        \item stabilizza la ferrite
        \item rallenta la traformazione eutettoidica
        \item accelera la trasformazione eutettoidica
        \item destabilizza la ferrite
    \end{enumerate}
    \item Il rinvenimento
    \begin{enumerate}
        \item aumenta la durezza
        \item aumenta la tenacità
        \item stabilizza la martensite
        \item stabilizza l'austenite
        \item si effettua dopo la normalizzazione
    \end{enumerate}
    \item Le leghe di alluminio
    \begin{enumerate}
        \item si saldano più facilmente
        \item hanno conducibilità termica inferiore agli acciai
        \item hanno bassa capacità termica
        \item il prezzo elevato è dovuto alla scarsità in natura
        \item il prezzo elevato è dovuto al processo di fabbricazione
    \end{enumerate}
    \item L’energia libera di un sistema liquido è minore di quella dello stesso sistema allo stato solido:
    \begin{enumerate}
        \item sempre
        \item sopra la temperatura di solidificazione
        \item sotto la temperatura di fusione
        \item all’equilibrio delle 2 fasi
        \item mai
    \end{enumerate}
    \item L'energia libera di formazione del biossido di carbonio:
    \begin{enumerate}
        \item aumenta all’aumentare della temperatura
        \item diminuisce all’aumentare della temperatura
        \item è indipendente dalla temperatura
        \item cresce, raggiunge un massimo e poi diminuisce all’aumentare della temperatura
        \item diminuisce, raggiunge un minimo e poi cresce all’aumentare della temperatura
    \end{enumerate}\item In un altoforno per ogni tonnellata di minerale introdotta, viene prodotto:
    \begin{enumerate}
        \item s T di \mathtext{CaCO_3}
        \item 2 T di coke
        \item 2 T di aria
        \item 1T di aria
        \item 1 T di coke
    \end{enumerate}
    \item In un altoforno per ogni tonnellata di minerale introdotta, viene prodotto:
    \begin{enumerate}
        \item 2 T di acciaio
        \item 2 T di ghisa
        \item 2 T di polvere
        \item 0,5 T di acciaio
        \item 0,5 T di ghisa
    \end{enumerate}
    \item I laminati a freddo destinati a ricottura statica:
    \begin{enumerate}
        \item devono essere laminati a temperatura più bassa possibile
        \item devono essere avvolti a temperatura più alta possibile
        \item non vengono influenzati dalla temperatura di laminazione
        \item devono essere avvolti a temperatura più bassa possibile
        \item non vengono influenzati dalla temperatura di avvolgimento
    \end{enumerate}
    \item La presenza del AlN in un acciaio da profondo stampaggio influenza:
    \begin{enumerate}
        \item il coefficiente di incrudimento
        \item il coefficiente d’anisotropia
        \item l’allungamento a carico max
        \item non ha influenza se non per essere un sottoprodotto della disossidazione
        \item la strizione
    \end{enumerate}
    \item Negli acciai trip il rafforzamento è dato da:
    \begin{enumerate}
        \item trasformazione alla martensite
        \item trasformazione dell'austenite
        \item trasformazione della ferrite
        \item trasformazione della cementite
        \item traformazione della perlite
    \end{enumerate}
    \item Come mezzo temprante, l'acqua:
    \begin{enumerate}
        \item Ha elevata V300 ed elevata V700
        \item  Ha bassa V300 ed elevata V700
        \item  Ha elevata V300 ed bassa V700 
        \item  Ha elevata V700
        \item  Ha bassa V300 ed bassa V700
    \end{enumerate}
    \item In che mezzo bisogna temprare un acciaio C40 (acciaio al carbonio con 0,40\%C)?
    \begin{enumerate}
        \item olio
        \item acqua 
        \item bagno salino
        \item gas sotto pressione
        \item idrocarburi
    \end{enumerate}
    \item La temperatura di solubilizzazione delle leghe di alluminio:
    \begin{enumerate}
        \item è >1200°C
        \item è <100°C
        \item è di 900°C
        \item è di 300°C
        \item è di 700°C 
    \end{enumerate}
    \item Tipicamente, lo strato interessato dalla cementazione è di:
    \begin{enumerate}
        \item qualche millimetro
        \item qualche micron
        \item qualche decimo di millimetro 
        \item qualche centimetro
        \item qualche nanometro
    \end{enumerate}
    \item Tipicamente, lo strato interessato dalla nitrurazione è di:
    \begin{enumerate}
        \item qualche millimetro
        \item qualche micron 
        \item qualche decimo di millimetro 
        \item qualche centimetro
        \item qualche nanometro
    \end{enumerate}
    \item Nella tempra superficiale per induzione, un aumento di frequenza:
    \begin{enumerate}
        \item aumenta lo spessore della zona interessata 
        \item stabilizza la martensite
        \item stabilizza l'austenite
        \item diminuisce lo spessore della zone interessata 
        \item aumenta la percentuale di carbonio nel reticolo
    \end{enumerate}
    \item Qual è il meccanismo di rafforzamento degli acciai CD
    \begin{enumerate}
        \item soluzione solida interstiziale
        \item precipitazione
        \item soluzione solida sostituzionale
        \item affinamento del grano
        \item incrudimento 
    \end{enumerate}
    \item La presenza di Mo negli acciai inossidabili ausenitici serve a:
    \begin{enumerate}
        \item limitare le dimensioni del grano
        \item favorire l'effetto del pitting
        \item contrastare l'effetto del pitting
        \item favorire la sensibilizzazione
        \item aumentare la resistenza alla corrosione alle basse temperature
    \end{enumerate}
     \item I duplex sono acciai:
    \begin{enumerate}
        \item altoresistenziali
        \item impiegati in campo automobilistico per alleggerire i pesi
        \item ferritico martensitici
        \item impiegati per resistere alla corrosione sotto sforzo
        \item austeno martensitici
    \end{enumerate}
     \item Nella prova Jominy il mezzo raffreddante è:
    \begin{enumerate}
        \item dipende dal tipo di acciaio
        \item olio
        \item acqua OK
        \item aria calma
        \item gas in pressione
    \end{enumerate}
     \item Un acciaio con tenore di carbonio 0,60\% dopo ricottura a temperatura ambiente è costituito da:
    \begin{enumerate}
        \item perlite
        \item 80\% perlite 20\% ferrite
        \item 75\% perlite 25\% ferrite
        \item 75\% perlite 25\% cementite
        \item 25\% perlite 75\% ferrite
    \end{enumerate}
     \item La ricottura dei lingotti avviene a:
    \begin{enumerate}
        \item 180°C per 12 ore
        \item 500°C per 1 ora
        \item 723°C per 12 ore
        \item 50°C sopra la temperatura di austenizzazione
        \item oltre i 1000°C per 100 ore
    \end{enumerate}
     \item L'infragilimento da idrogeno avviene soprattutto in:
    \begin{enumerate}
        \item tutti gli acciai ferritici
        \item tutti gli acciai auastenitici
        \item acciai ferritici con elevate caratteristiche meccaniche
        \item acciai ferritici con scarse caratteristiche meccaniche
        \item acciai austenitici con elevate caratteristiche meccaniche
    \end{enumerate}
    
     \item Acciaio A con 0,3\% di C     acciaio B con 0,3\% C e 2\% Cr:
    \begin{enumerate}
        \item La curva Jominy di A sta sempre sotto quella di B
        \item Le curve Jominy di A e di B partono nello stesso punto
        \item La durezza nel primo punto dell'acciaio B è maggiore di quella dell'acciaio A
        \item La durezza nel primo punto dell'acciaio A è maggiore di quella dell'acciaio B
        \item La curva Jominy di A sta sopra quella di B per un tratto iniziale
    \end{enumerate}
     \item Sono suscettibili all'invecchiamento le leghe della serie
    \begin{enumerate}
        \item 1000
        \item 3000
        \item 4000
        \item 5000
        \item 6000 OK
    \end{enumerate}
\end{enumerate}

\footnote[0]{Risposte: 1.a, 2.d, 3.e, 4.d, 5.a, 6.c, 7.a, 8.a, 9.b, 10.d, 11.d, 12.c, 13.e, 14.c, 15.b, 16.d, 17.c, 18.a, 19.c, 20.b, 21.c, 22.e, 23.e, 24.d, 25.d, 26.e, 27.a, 28.e, 29.d, 30.b, 31.c, 32.a, 33.b, 34.e, 35.a, 36.a, 37.c, 38.e, 39.d, 40.a, 41.b, 42.c, 43.b, 44.e, 45.c, 46.b, 47.d, 48.e.}
