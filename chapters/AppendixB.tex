\setchapterstyle{lines}
\labpage{appendix}

\chapter{Sigle degli acciai}\footnote{Da https://it.wikipedia.org/wiki/Acciaio\_(sistemi\_di\_designazione) consultato il 02/02/2021}

Alcuni simboli e sigle degli acciai sono sparsi all'interno dei capitoli. Vengono qui raccolte nella speranza (vana)\sidenote{Vanissima}\footnote{Veramente vana vana vana} che non vengano chieste in sede d'esame. Sono evidenziate sigle già presenti all'interno di questi appunti o già capitate all'esame.

La normativa UNI EN 10027/1 designa gli acciai secondo 2 gruppi: (vedi anche UNI 10440)
\begin{itemize}
\item 1º gruppo: designazione in base all'impiego ed alle caratteristiche meccaniche o fisiche;
\item 2º gruppo: designazione in base alla composizione chimica (suddivisi in quattro sottogruppi).
\end{itemize}
\subsection{Primo gruppo}
\subsubsection{Primo simbolo}
\begin{itemize}
    \item B - acciai per calcestruzzo armato ordinario;
    \item \textbf{C - acciai non legati al carbonio};
    \item \textbf{D - acciai prodotti piani per formatura a freddo};
    \item E - acciai per costruzioni meccaniche;
    \item G - acciai da getti (di acciaio grezzo, da fonderia);
    \item H - acciai ad alta resistenza per prodotti piani laminati a freddo e per imbutitura a freddo;
    \item HS - acciai rapidi;
    \item L - acciai per tubi di condutture;
    \item M - acciai magnetici;
    \item P - acciai per impieghi sotto pressione;
    \item \textbf{R - acciai per rotaie (basso coefficiente di dilatazione)};
    \item S - acciai per impieghi strutturali (carpenterie metalliche);
    \item T - acciai per banda nera, stagnata e cromata (per imballaggi);
    \item \textbf{X - acciai legati (esempio: acciai inox)};
    \item Y - acciai per calcestruzzo armato precompresso.
\end{itemize}

\subsubsection{Secondo simbolo}

Alcuni dei simboli precedenti sono seguiti da un simbolo che ne identifica alcune caratteristiche meccaniche (qui omesse perché sarebbe troppo anche per Giorgio). Sono poi poste delle ulteriori lettere che ne danno delle caratteristiche fisiche:
\begin{itemize}
    \item D
    \begin{itemize}
         \item \textbf{lettera C per prodotti laminati a freddo};
        \item lettera D per prodotti laminati a caldo ma destinati alla formatura a freddo;
        \item lettera X per prodotti con stato di laminazione non specificato.
    \end{itemize}
    \item M
    \begin{itemize}
        \item seguita da: 1] numero pari a 100 volte la perdita specifica massima in W/kg (induzione magnetica a 50 Hz); 2] numero pari a 100 volte lo spessore nominale del prodotto in mm; 3] ulteriori simboli che specificano il tipo di acciaio magnetico 
    \end{itemize}
\end{itemize}

\subsubsection{Altri simboli}
Seguono per alcuni acciai alcune altre caratteristiche, seguite a loro volta da:
\begin{itemize}
    \item simbolo AR (As Rolled) se grezzo di laminazione;
 \item simbolo C per formatura speciale a freddo;
 \item simbolo D per zincatura;
 \item simbolo E per smaltatura;
 \item simbolo G1 per acciaio effervescente;
 \item simbolo G2 per acciaio calmato;
 \item simbolo G3 per stato di fornitura opzionale (stato di disossidazione non definito);
 \item simbolo G4 per stato di fornitura a descrizione del produttore;
 \item simbolo H per profilo cavo;
 \item simbolo KU per acciaio per utensili;
 \item simbolo L per bassa temperatura;
\item simbolo M per laminazione termomeccanica;
 \item simbolo N per laminazione di normalizzazione;
 \item simbolo O per offshore;
 \item \textbf{simbolo Q per acciaio bonificato (ad alto limite di snervamento)};
 \item simbolo S per costruzioni navali;
 \item simbolo T per tubi;
\item simbolo W per acciaio resistente alla corrosione atmosferica;
\end{itemize}

\subsection{Secondo gruppo}
La designazione varia a seconda del tipo di acciaio e della percentuale degli elementi di lega:

\subsubsection{Acciai non legati con tenore di manganese < 1\% (tranne gli acciai non legati per lavorazioni meccaniche ad alta velocità, detti anche “automatici”)}

Lettera C seguita da un numero pari a 100 volte il tenore percentuale di carbonio medio prescritto. Non legato non significa che non abbia altri elementi in lega, ma che le loro percentuali sono trascurabili.

esempio:
\begin{itemize}
    \item C35 acciaio del 2º gruppo, non legato, con 0,35\% C. (acciaio dolce);
\end{itemize}

\subsubsection{Acciai non legati con tenore di manganese $\ge$ 1\% e acciai non legati per lavorazioni meccaniche ad alta velocità (“automatici”) e acciai legati (non rapidi) con tenore di ciascun elemento di lega < 5\% (acciai bassolegati o debolmente legati)}

Numero pari a 100 volte il tenore percentuale di carbonio medio prescritto, seguito dagli elementi chimici di lega presenti in ordine decrescente di concentrazione (per tenori uguali si usa l'ordine alfabetico), seguiti dai rispettivi valori delle loro concentrazioni in percentuale, ma da correggere con coefficienti correttivi, e separati da trattino.

Coefficienti correttivi:
\begin{itemize}
    \item 4 per cobalto (Co), cromo (Cr), manganese (Mn), nichel (Ni), silicio (Si), tungsteno (W);
    \item 10 per alluminio (Al), berillio (Be), rame (Cu), molibdeno (Mo), niobio (Nb), piombo (Pb), tantalio (Ta), titanio (Ti), vanadio (V), zirconio (Zr);
    \item 100 per cesio (Ce), azoto (N), fosforo (P), zolfo (S);
    \item 1000 per il boro (B)
\end{itemize}

esempio: 
\begin{itemize}
    \item 30NiCrMo4-2 acciaio del 2º gruppo, debolmente legato, con 0,30\% C, 1\% Ni, 0,5\% Cr, \% Mo non dichiarata. [1\% Ni perché 4/4=1 ; 0,5\% Cr perché 2/4=0,5];
\end{itemize}

 \subsubsection{acciai legati (non rapidi) con tenore di almeno uno degli elementi di lega $\ge$ 5\% (acciai altolegati o fortemente legati o inox)}

Lettera X seguita da un numero pari a 100 volte il tenore percentuale di carbonio medio prescritto, seguito dagli elementi chimici di lega presenti in ordine decrescente di concentrazione (per tenori uguali si usa l'ordine alfabetico), seguiti dai rispettivi valori delle loro concentrazioni in percentuale (arrotondati al numero intero più vicino ma non da correggere coi coefficienti), e separati da trattino.

esempio:
\begin{itemize}
    \item X4CrNiMo17-12-2 acciaio del 2º gruppo, fortemente legato, con 0,04\% C, 17\% Cr, 12\% Ni, 2\% Mo.
\end{itemize}

\subsubsection{acciai rapidi}

Sono indicati con le lettere HS seguite dal tenore percentuale medio (arrotondato al numero intero più vicino) di tungsteno (\% W), molibdeno (\% Mo), vanadio (\% V) e cobalto (\% Co). Non è indicata la \% C.

esempio:
\begin{itemize}
    \item HSHS18-0-1 acciaio del 2º gruppo, rapido, con 18\% W, 1\% V.
\end{itemize}